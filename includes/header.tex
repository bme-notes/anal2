%% BME-Notes jegyzethez header
%% Ha nem érted, mi történik itt, akkor inkább ne változtasd meg!
%% A fájlok fordításához XeLaTeX-et kell használni
\documentclass[dvipsnames]{article}
\usepackage{lmodern}
\usepackage{amssymb}
\usepackage{amsmath}
\usepackage{polyglossia}
\usepackage{listings}
\usepackage{tcolorbox}
\usepackage{etoolbox}
\PassOptionsToPackage{dvipsnames}{xcolor}
\usepackage[dvipsnames]{xcolor}
\usepackage{setspace}
\usepackage{framed}
\usepackage[a4paper,margin=2.5cm]{geometry}
\usepackage{fancyhdr}
\pagestyle{fancy}
\usepackage[hidelinks]{hyperref}
\definecolor{shadecolor}{HTML}{eeeeee} %Kindle-optimized color
\setcounter{secnumdepth}{0} %this automagically removes extra numbering toc and sections
\renewcommand{\contentsname}{Tartalomjegyzék}
\newtoks\cim
\newtoks\szerzo
\newtoks\segitettek
\newtoks\datum

\newcommand{\ujfejezet}[1]{\newpage \input{./fejezetek/#1.tex}}
\newenvironment{tetel}[1]{\begin{framed}\noindent\ignorespaces\textbf{\large Tétel: #1}\normalsize\\}{\end{framed}\ignorespacesafterend}
\newenvironment{definicio}[1]{\begin{shaded}\noindent\ignorespaces\textbf{\large Definíció: #1}\normalsize\\}{\end{shaded}\ignorespacesafterend}
\newenvironment{bizonyitas}[1]{\begin{leftbar}\noindent\ignorespaces\textbf{\large Bizonyítás: #1}\normalsize\\}{\end{leftbar}\ignorespacesafterend}

\definecolor{darkgreen}{HTML}{098905}

\newcommand{\n}{\ensuremath{\textcolor{blue}{n}}}
\renewcommand{\k}{\ensuremath{\textcolor{PineGreen}{k}}}
\newcommand{\x}{\ensuremath{\textcolor{red}{x}}}
\renewcommand{\u}{\ensuremath{\textcolor{darkgreen}{u}}}
\renewcommand{\c}{\ensuremath{\textcolor{Sepia}{c}}}
\newcommand{\q}{\ensuremath{\textcolor{SkyBlue}{q}}}
\newcommand{\sumi}{\ensuremath{\sum_{\n= 0}^{\infty}}}

\title{\huge\textsc{\the\cim}}
\author{\the\szerzo}
\date{\the\datum}
